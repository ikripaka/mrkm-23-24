%!TEX root = ../thesis.tex
% створюємо розділ

\section{Основні задачі, що виникають при програмній реалізації криптосистем на токенах та сматр картках}

На останок можна коротко розказати про основні проблеми, що можуть виникнути при реалізації криптосистем на токенах та сматри картках. Відповідно до проведеного дослідження проблеми можна класифікувати наступним чином.

\begin{itemize}
    \item \textbf{Генератор псевдо випадкових чисел.} У токенах дуже важливо мати надійне джерело ентропії для генерування ключових пар.
    \item \textbf{Обмежена кількість пам'яті та обчислювальні можливості.} Так як токен чи смарт-картка мають досить малий форм-фактор, то до них не вдастся приєднати більшу кількість пам'яті. Тоді коли на комп'ютері можна сказаати <<необмежені>> можливості для обчислень, то у токенах розробникам приходиться максимально оптимізовувати їх реалізації. 
\end{itemize}