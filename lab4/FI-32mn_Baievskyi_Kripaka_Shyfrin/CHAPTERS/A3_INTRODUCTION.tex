%!TEX root = ../thesis.tex
% створюємо вступ

\section{Мета практикуму}

Дослідити основні задачі, що виникають при програмній реалізації криптосистем. Запропонувати методи вирішення задачі контролю доступу до ключової інформації, що зберігається в оперативній пам’яті ЕОМ для різних (обраних) операційних систем. Запропонувати методи вирішення задачі контролю правильності функціонування програми криптографічної обробки інформації. Порівняти з точки зору вирішення цих задач інтерфейси Crypto API, PKCS11. Дослідити основні задачі, що виникають при програмній реалізації криптосистем. Запропонувати методи вирішення задачі контролю доступу до ключової інформації, що зберігається в оперативній пам’яті ЕОМ для різних (обраних) операційних систем. Запропонувати методи вирішення задачі контролю правильності функціонування програми криптографічної обробки інформації. Порівняти з точки зору вирішення цих задач інтерфейси Crypto API, PKCS11.

\subsection{Постановка задачі та варіант}
\begin{tabularx}{\textwidth}{X|X}
	\textbf{Треба виконати} & \textbf{Зроблено} \\
	Дослідити основні задачі, що виникають при програмній реалізації криптосистем на токенах та сматр картках & \checkmark \\
	% Дослідити та запропонувати методи вирішення задачі контролю доступу до ключової інформації & \checkmark \\
 %    Дослідити та запропонувати методи вирішення задачі контролю правильності функціонування програми криптографічної обробки інформації & \checkmark \\
	Дослідити різницю Crypto API та PKCS11 інтерфейсів & \checkmark \\
    Навести приклад роботи із PKCS11 інтерфейсом із токеном від Yubikey & \checkmark \\
\end{tabularx}

\section{Хід роботи/Опис труднощів}
    На початку роботи над практикума вибрати варіант 1А, та далі продовжували роботу над ними. Згідно вибраного варіанту у даній роботі буде розглянуто методи вирішення на токенах та смарт пристроях. Під час виконання звіту виникала лише одна часова складність.