%!TEX root = ../thesis.tex
% створюємо розділ
\chapter{Способи розділення секрету в базових протоколах}
\label{chap:review2}
 % \protect\\

\section{Розділення секрету без Трента}
Банк хоче, щоб його сховище відкрилося, лише якщо троє з п’яти співробітників введуть свої ключі. Це звучить як базова (3,5)-порогова схема, але є підступ. Ніхто не повинен знати весь секрет. Немає Трента, щоб розділити секрет на п’ять частин. Існують протоколи, за якими п’ятеро офіцерів можуть створити секрет і кожен отримає шматочок, так що жоден з офіцерів не дізнається секрету, доки всі не відновлять його \cite{appCrypto}.

\section{Розширені порогові схеми розділення секрету}
Попередні приклади ілюструють лише найпростіші порогові схеми. Далі буде використано алгоритм Шаміра, хоча будь-який з інших буде працювати.

Щоб створити схему, у якій одна людина важливіша за іншу, дайте цій людині більше тіней. Якщо для відтворення секрету потрібно п’ять тіней і одна людина має три тіні, а всі інші мають лише одну, тоді ця особа та ще двоє людей можуть відтворити секрет. Без цієї людини знадобиться п’ять, щоб відтворити секрет.
Двоє або більше людей можуть отримати кілька тіней. У кожної людини може бути різна кількість тіней. Незалежно від того, як тіні розподілені, будь-яка $m$ з них може бути використана для відновлення секрету. Хтось із $m-1$ тіней, будь то одна особа чи ціла кімната людей, не може цього зробити.
Крім цього є варіанти, коли приймають участь дві ворожі делегації. 
Ви можете поділитися секретом, щоб відтворити секрет двом особам із 7 у делегації A та 3 особам із 12 у делегації B. Складіть поліном третього ступеня, який є добутком лінійного виразу на квадратичний вираз. Дайте кожному з делегації A тінь, яка є результатом оцінки лінійного рівняння; дайте кожному з делегації B тінь, яка є оцінкою квадратного рівняння.
Будь-які дві тіні від Делегації А можна використати для реконструкції лінійного рівняння, але незалежно від того, скільки інших тіней має група, вони не можуть отримати жодної інформації про секрет. Те саме стосується делегації B: вони можуть зібрати три тіні разом, щоб реконструювати квадратне рівняння, але вони не можуть отримати жодної

\section{Розділення секрету з шахраями}
Існує багато способів обдурити за допомогою порогової схеми. Оскільки при розподілі секрету гарантується, що якщо будь-який учасник отримає секрет, інші учасники також отримають. Таким чином в схемі можуть бути шахраї, приховані в учасниках, які можуть загрожувати компроментації секрету.
Для боротьби з шахраями використовують спеціальну схему розділення секрету з шахраями. Розглянемо її далі \cite{sssCheaters}.

Алгоритм "Розділення секрету з шахраями" модифікує стандартну (m,~n)-порогову схему для виявлення шахраїв. Це буде продемонстровано за допомогою схеми Лагранжа, хоча вона також працює з іншими.

Виберіть просте число $p$, яке одночасно більше за $n$ і більше за $\frac{(s-1)(m-1)}{e+m}$, де $s$ -- найбільший можливий секрет, а $e$ -- ймовірність успішного шахрайства. Ви можете зробити $e$ настільки малим, скільки хочете; це лише ускладнює обчислення. Побудуйте свої тіні, як і раніше, за винятком використання замість 1, 2, 3, . . . , $n$ для $x_i$; оберіть випадкові числа від $1$ до $p-1$ для $x_i$.

Тепер, коли шахрай пробирається на таємну нараду з реконструкції зі своєю фальшивою часткою, велика ймовірність того, що його частка буде неможливою. Неможлива таємниця -- це, звичайно, фальшива таємниця.

На жаль, хоча шахрая і викривають, він все одно дізнається секрет (припускаючи, що є $m$ інших дійсних спільних ресурсів). Інший протокол, запобігає цьому. Основна ідея полягає в тому, щоб мати серію з $k$ секретів, щоб жоден з учасників не знав заздалегідь, який правильний. Кожен секрет більший за попередній, за винятком справжнього секрету. Учасники об’єднують свої тіні, щоб генерувати один секрет за іншим, поки не створять секрет, менший за попередній. Це правильний варіант.

Ця схема викриє шахраїв завчасно, до того, як буде згенеровано секрет. Також бувають ускладнення, коли учасники доставляють свої тіні по одному.

\chapconclude{\ref{chap:review2}}

В даному розділі було розглянуто різні способи розділення секрету в базових протоколах. А конкреніше -- метод розділення секрету без Трента, який не потребує третьої сторони (Трента) для розподілу секрету; розширені порогові схеми розділення секрету, які дозволяють учасникам динамічно приєднуватися та виходити з групи, а також змінювати поріг відновлення секрету; схеми розділення секрету з шахраями, які стійкі до атак зловмисників, що можуть отримати доступ до частин секрету. Кожен з цих методів має свої переваги та недоліки. Вибір найкращого методу залежить від конкретних потреб та вимог до безпеки.