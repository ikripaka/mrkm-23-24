%!TEX root = ../thesis.tex
% створюємо вступ
В епоху неспинно зростаючого розвитку технологій у сучасному світі, питання безпеки даних і персональної інформації користувачів стає невід'ємною частиною розробки сучасних продуктів та екосистем. Еволюція типів з'єднання пристроїв та підвищення рівню різноманітністі атак і шкідливого програмного забезпечення роблять захист інформації критично важливим.

В основі безпечних систем лежать випадкові числа, які відіграють ключову роль у криптографічних операціях. Ці числа використовуються для створення надійних ключів шифрування, встановлення початкових значень і лічильників, а також параметрів протоколів. Ступінь випадковості цих чисел безпосередньо впливає на безпеку системи, і будь-яка слабкість у процесі їхньої генерації може призвести до відкриття дверей для атак, які можуть скомпрометувати ключі, перехопити дані та в кінцевому підсумку зламати пристрої та їхні комунікації.

Розробка генераторів істинних випадкових чисел, що забезпечують стабільно високу якість ентропії під час зміни процесів, температури, напруги і частоти, дуже складний процес. Для забезпечення найвищої якості міжнародні органи стандартизації розробили критерії, що дають змогу підтвердити істинно випадковий характер ГВЧ перевіреним і статистично суворим чином.

У роботі розглядається важливість TRNG \cite{synopsys2019true}, які використовуються для генерації ПВЧ для інтелектуальних карт \cite{pannetrat2015true}, токенів та смартфонів, описуються їхні основні компоненти та характеристики. Також наводиться приклад статті із дослідженням генераторів певної структури на основі TRNG \cite{pseudorandomSmartCards}.